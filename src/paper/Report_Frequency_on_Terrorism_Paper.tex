\documentclass[11pt, a4paper, leqno]{article}
\usepackage{a4wide}
\usepackage[T1]{fontenc}
\usepackage[utf8]{inputenc}
\usepackage{float, afterpage, rotating, graphicx}
\usepackage{epstopdf}
\usepackage{longtable, booktabs, tabularx}
\usepackage{fancyvrb, moreverb, relsize}
\usepackage{eurosym, calc}
% \usepackage{chngcntr}
\usepackage{amsmath, amssymb, amsfonts, amsthm, bm}
\usepackage{caption}
\usepackage{mdwlist}
\usepackage{xfrac}
\usepackage{setspace}
\usepackage{xcolor}
\usepackage{subcaption}
\usepackage{minibox}
% \usepackage{pdf14} % Enable for Manuscriptcentral -- can't handle pdf 1.5
% \usepackage{endfloat} % Enable to move tables / figures to the end. Useful for some submissions.


\usepackage[
natbib=true,
bibencoding=inputenc,
bibstyle=authoryear-ibid,
citestyle=authoryear-comp,
maxcitenames=3,
maxbibnames=10,
useprefix=false,
sortcites=true,
backend=biber
]{biblatex}
\AtBeginDocument{\toggletrue{blx@useprefix}}
\AtBeginBibliography{\togglefalse{blx@useprefix}}
\setlength{\bibitemsep}{1.5ex}
\addbibresource{refs.bib}





\usepackage[unicode=true]{hyperref}
\hypersetup{
	colorlinks=true,
	linkcolor=black,
	anchorcolor=black,
	citecolor=black,
	filecolor=black,
	menucolor=black,
	runcolor=black,
	urlcolor=black
}


\widowpenalty=10000
\clubpenalty=10000

\setlength{\parskip}{1ex}
\setlength{\parindent}{0ex}
\setstretch{1.5}


\begin{document}
	
	\title{Assignment 6\thanks{Laurenz Guenther, BGSE. Email: \href{mailto:s3laguen@uni-bonn.de}{\nolinkurl{s3laguen [at] uni-bonn [dot] de}}.}}
	
	\author{Laurenz Guenther}
	
	\date{
		{\bf Preliminary -- please do not quote} 
		\\[1ex] 
		\today
	}
	
	\maketitle
	
	
	\begin{abstract}
		The goal of this research project is to evaluate how biased newspaper coverage of terrorist attacks is. The current manuscript is a very early version. It contains the frequency of reporting on islamist terrorist attacks by the most popular german newspaper over the period of 2006-2019. 
	\end{abstract}
	\clearpage
	
	\section{Introduction} % (fold)
	\label{sec:introduction}



Interest in biased media coverage has recently spiked, triggered by several surveys indicating a loss of trust of households in the media [CITE]. 

This lost in trust also has a ploitical component, as it is often used as argument by extremist groups as a justification for widerstand against the political ordnung [CITE], possibly leading to serious verwerfungen. It is hence important to understand in which sense the media is biased and why it is perceived as biased.

Observation suggests that both, left, islamists and right wing extremists regularly claim the media biased against them[CITE]. A commoon argument is that ill-doings of the own group are reported on more often than on similar doings by the other side. To better understand how this perception compares to reality, I want to compute the actual probability that several kinds of extremism are reported on if casdulties are similar bur prepetuators differ regarding their ideology.



For instance, one might suspect that terrorist attacks are reported more frequently, all else equal, if they are committed by islamists than if they are committed by non-islamists. 

Figure ... does a first attempt .....







RELATED LIETRATU
Accordingly, several measures have put forward in past years to measure media bias. ..... [CITE]








 \footnote{This paper was generated using \citet{GaudeckerEconProjectTemplates}.}
%\input{formulas/decision_rule}

\begin{figure}
   \caption{A simple line-chart showing the number of articles in the Bildzeitung on islamist terrorist attacks per year.}
  
 \includegraphics[width=\textwidth]{../../out/figures/line_chart.pdf}

\end{figure}


 section introduction (end)




\setstretch{1}
%\printbibliography
\setstretch{1.5}




% \appendix

% The chngctr package is needed for the following lines.
% \counterwithin{table}{section}
% \counterwithin{figure}{section}

\end{document}
