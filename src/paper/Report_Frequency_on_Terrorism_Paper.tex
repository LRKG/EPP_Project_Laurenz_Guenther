\documentclass[11pt, a4paper, leqno]{article}
\usepackage{a4wide}
\usepackage[T1]{fontenc}
\usepackage[utf8]{inputenc}
\usepackage{float, afterpage, rotating, graphicx}
\usepackage{epstopdf}
\usepackage{longtable, booktabs, tabularx}
\usepackage{fancyvrb, moreverb, relsize}
\usepackage{eurosym, calc}
% \usepackage{chngcntr}
\usepackage{amsmath, amssymb, amsfonts, amsthm, bm}
\usepackage{caption}
\usepackage{mdwlist}
\usepackage{xfrac}
\usepackage{setspace}
\usepackage{xcolor}
\usepackage{subcaption}
\usepackage{minibox}
% \usepackage{pdf14} % Enable for Manuscriptcentral -- can't handle pdf 1.5
% \usepackage{endfloat} % Enable to move tables / figures to the end. Useful for some submissions.


\usepackage[
natbib=true,
bibencoding=inputenc,
bibstyle=authoryear-ibid,
citestyle=authoryear-comp,
maxcitenames=3,
maxbibnames=10,
useprefix=false,
sortcites=true,
backend=biber
]{biblatex}
\AtBeginDocument{\toggletrue{blx@useprefix}}
\AtBeginBibliography{\togglefalse{blx@useprefix}}
\setlength{\bibitemsep}{1.5ex}
\addbibresource{refs.bib}





\usepackage[unicode=true]{hyperref}
\hypersetup{
	colorlinks=true,
	linkcolor=black,
	anchorcolor=black,
	citecolor=black,
	filecolor=black,
	menucolor=black,
	runcolor=black,
	urlcolor=black
}


\widowpenalty=10000
\clubpenalty=10000

\setlength{\parskip}{1ex}
\setlength{\parindent}{0ex}
\setstretch{1.5}


\begin{document}
	
	\title{Media Bias in Newspaper Reports on Terrorism\thanks{Laurenz Guenther, BGSE. Email: \href{mailto:s3laguen@uni-bonn.de}{\nolinkurl{s3laguen [at] uni-bonn [dot] de}}.}}
	
	\author{Laurenz Guenther}
	
	\date{
		{\bf Preliminary -- please do not quote} 
		\\[1ex] 
		\today
	}
	
	\maketitle
	
	
	\begin{abstract}
		The goal of this research project is to evaluate how biased newspaper coverage of terrorist attacks is. The current manuscript is a very early version. It contains the frequency of reporting on islamist terrorist attacks by the most popular German newspaper between 2006 and 2019. 
	\end{abstract}
	\clearpage
	
	\section{Introduction} % (fold)
	\label{sec:introduction}



Interest in biased media coverage has recently spiked, triggered by several surveys indicating a loss of trust of households in the media \cite{puglisi2015empirical}. 

This lost in trust also has a political component, as it is often used as argument by extremist groups as a justification for attacks against the political establishment (\cite{schellenberg2016lugenpresse}). It is hence important to why the media is perceived as biased.

Observation suggests that both, left, Islamists and right wing extremists regularly claim the media biased against them (\cite{groseclose2005measure}). A common argument is that ill-doings of the own group are reported on more often than on similar doings by the other side. To better understand how this perception compares to reality, I want to compute the actual probability that several kinds of extremism are reported on if casualties are similar bur perpetrators differ regarding their ideology.



Figure 1 does a first attempt to answering this question by plotting the number of articles written in the Bildzeitung, the most popular German newspaper, for each year between 2006 and 2019. The underlying algorithm uses a simple Bayesian Algorithm to learn to distinguish articles about Islamist terrorism from other articles and then classifies all articles from the online archive of the Bildzeitung accordingly.













 \footnote{This paper was generated using \citet{GaudeckerEconProjectTemplates}.}
%\input{formulas/decision_rule}

\begin{figure}
   \caption{A simple line-chart showing the number of articles in the Bildzeitung on islamist terrorist attacks per year.}
  
 \includegraphics[width=\textwidth]{../../out/figures/line_chart.pdf}

\end{figure}


 section introduction (end)




\setstretch{1}
%\printbibliography
\setstretch{1.5}




% \appendix

% The chngctr package is needed for the following lines.
% \counterwithin{table}{section}
% \counterwithin{figure}{section}

\end{document}
